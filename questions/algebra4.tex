% this is a test
% assignment 1
\begin{problem}
Let \( * \) be a binary law on \( (\Z, +) \) such that \( (\Z, +, *) \) is a unital ring.
Show that \( (\Z, +, *) \) is isomorphic to the ring of integers.
\end{problem}

\begin{problem}
Let \( R \) be a unital commutative ring.
Let \( I \ideal R \) be an ideal and let
\[ \sqrt I = \set{a \in R \given \exists n > 0,\ a^n \in I}. \]
\begin{enumerate}[1]
\item
Show that \( \sqrt I \) is an ideal of \( R \).
\item
Show that \( I \subset \sqrt I \), that if \( I \subset J \) then \( \sqrt I \subset \sqrt J \), and that \( \sqrt{\sqrt I} = \sqrt I \).
\item
Show that if \( n = p_1^{k_1} \cdots p_r^{k_r} \) is an integer, then the radical of \( (n) \) is the ideal generated by \( p_1 \cdots p_r \).
In particular, for any \( n, m \), we have \( \sqrt{(n)(m)} = \sqrt{(n) \cap (m)} \) since both ideals involve the same primes.
\item
Show that if \( I, J \ideal R \), then \( \sqrt{IJ} = \sqrt{I \cap J} \).
\end{enumerate}
\end{problem}

\begin{problem}
Let \( \mathbb H = \set{a + bi + ci + dk \given a, b, c, d \in \R} \) denote the (division) ring of Hamilton quaternions.
Show that the set
\[ R = \set{a + bi + cj + dk \given a,b,d,c \in \Z \text{ or } a,b,c,d \in \frac 1 2 + \Z} \]
is a subring of \( \mathbb H \).
\end{problem}

\begin{problem}
Let \( K \) be a field.
Characterize the units in the ring \( K[[X]] \) of formal power series.
\end{problem}

% assignment 2
\begin{problem}
Assume \( R \) is commutative. Let \( x \) be an indeterminate, let \( f(x) \) be a monic polynomial in \( R[x] \) of degree \( n \geq 1 \) and use the bar notation to denote passage to the quotient ring \( R[x] / (f(x)) \).
\begin{enumerate}[a]
\item
Show that every element of \( R[x] / (f(x)) \) is of the form  \( \overline{p(x)} \) for some polynomial \( p(x) \in R[x] \) of degree less than \( n \), i.e.
\[ R[x] / (f(x)) = \{ \overline{a_0} +  \overline{a_1x} + \dots +  \overline{a_{n-1}x^{n-1}} | a_0, a_1, \dots a_{n-1} \in R \}. \]
[If \(f(x) = x^n + b_{n-1}x^{n-1} + \dots + b_0\) then \(\overline{x^n} = \overline{-(b_{n-1}x^{n-1} + \dots + b_0)}\). Use this to reduce powers of \( \overline{x} \) in the quotient ring.]
\item
Prove that if \( p(x) \) and \( q(x) \) are distinct polynomials in \( R[x] \) which are both of degree less than \( n \), then \( \overline{p(x)} \neq \overline{q(x)} \). [Otherwise \( p(x) - q(x) \) is an \( R[x] \)-multiple of the monic polynomial \( f(x) \).]
\item
If \( f(x) = a(x)b(x) \) where both \( a(x) \) and \( b(x) \) have degree less than \( n \), prove that \( \overline{a(x)} \) is a zero divisor in \( R[x] / (f(x)) \).
\item
If \( f(x) = x^n - a \) for some nilpotent element \( a \in R \), prove that \( \overline{x} \) is nilpotent in \( R[x] / (f(x)) \).
\item
Let \( p \) be a prime, assume \( R = \F_p \) and \( f(x) = x^p - a \) for some \( a \in \F_p \). Prove that \( \overline{x-a} \) is nilpotent in \( R[x] / (f(x)) \).
\end{enumerate}
\end{problem}

\begin{problem}
Let \( R \) be a unital commutative ring. Show that \( R \setminus R^\times \) is an ideal if and only if \( R \) contains a unique maximal ideal.
\end{problem}

\begin{problem}
Provide examples (and supporting proofs) of (using commutative unital rings)
\begin{enumerate}[1]
\item
a nonzero prime ideal which is not maximal.
\item
a ring without nonzero nilpotent elements which is not an integral domain.
\item
an integral domain in which the intersection of maximal ideals is trivial.
\item
an integral domain in which the intersection of maximal ideals is non-trivial.
\end{enumerate}
\end{problem}

\begin{problem}
Let \(R\) be a unital commutative ring.
\begin{enumerate}[1]
\item
Show that if \( r \in R \) is nilpotent, then \( 1 - r \) is invertible.
\item
Show that if \( r \in R \) is nilpotent, then \( 1 + (r) \subset R^\times \).
\end{enumerate}
\end{problem}

\begin{problem}
Let \( p \) be a rational (i.e. usual) prime. Prove that if \( p \) is not a Gaussian prime, then it is the product of a Gaussian prime and its conjugate.
\end{problem}

% assignment 3
\begin{problem}
Let \( I \) be an ideal in a unital commutative ring \( R \).
\begin{enumerate}[1]
\item
Show that \( \sqrt{I} \) is the inverse image of the nilradical of \( R/I \).
\item
Show that \( \sqrt{I} \) is the intersection of prime ideals containing  \( I \).
\end{enumerate}
\end{problem}

\begin{problem}
In the ring \( \Q[X, Y] \), do the elements \( X \) and \( Y \) have a gcd? Is the ideal \( (X, Y) \) principal?
\end{problem}

\begin{problem}
Let \( R \) be a commutative unital ring. An ideal \( I \) in \( R \) is said to be irreducible if it cannot be expressed as the intersection of two strictly larger ideals:
\[ I = J \cap K \implies I = J \text{ or } I = K \]
\begin{enumerate}[1]
\item
Check that if \( I \) is a prime ideal, it is irreducible.
\item
Check that if \( p \) prime and \( n \geq 1 \), then \( (p^n) \) is irreducible.
\end{enumerate}
\end{problem}

\begin{problem}
\begin{enumerate}[1]
\item
Prove that \( \Z[\sqrt{2}] = \{a + b \sqrt{2} , a, b \in \Z \} \) is Euclidean.
\item
Prove that  \( \Z[\sqrt{2}] \) contains infinitely many units, of the form  \( \pm(1 + \sqrt{2})^n \).
\end{enumerate}
\end{problem}

% assignment 4
\begin{problem}
Let \( M \) be a module which admits a countable generating set. Show that the following are equivalent:
\begin{enumerate}[1]
\item
\( M \) is f.g.
\item
Every chain \( (M_n) \) of submodules (chain = nested sequence) of \( M \) such that \( M = \cup_n M_n \) stabilizes.
\item
For every surjective homomorphism \( F_A \twoheadrightarrow M \), where \( F_A \) is the free module on the countable set \( A \), there exists a finite subset \( B \subset A \) such that the restriction \( F_B \twoheadrightarrow M \) is surjective.
\end{enumerate}
\end{problem}

\begin{problem}
\begin{enumerate}[1]
\item
Show that a simple module over \( \Z \) is finite.
\item
Show that a simple module over \( K[x] \) (where \( K \) is a field) is finite dimensional.
(Solve this exercise in one stroke).
\end{enumerate}
\end{problem}

\begin{problem}
Let \( R \) be a Noetherian domain (i.e. an integral domain which is Noetherian). Prove that every \( r \in R \) can be decomposed as a product \( r = up_1 \dots p_n \) where the \( p_i\)'s are irreducible and \( u \) is a unit.
\end{problem}

\begin{problem}
Let \( \theta : M \to M \) be an \( R \)-module map. Suppose that \( M \) is Noetherian.
\begin{enumerate}[1]
\item
Show that \( \ker \theta^n = \ker \theta^{n+1} \) for all large integer \( n \).
\item
Show that \( \ker \theta^n \cap \mathrm{Im} \theta^n = 0 \) for some large integer \( n \).
\item
Show that if \( \theta \) is surjective, then it is injective.
\end{enumerate}
\end{problem}

\begin{problem}
Let \( R \) be an integral domain and \( M \) be a f.g. \( R \)-module. We define the rank of \( M \) to be the largest integer \( n \) such that \( M \) contains a free module \( F \isom R^n \). In class we showed that (if \( R \) is an integral domain and \( M \) is a f.g. \( R \)-module) there exists a free submodule \( F \subset M \) such that \( M/F \) is a torsion module. Show that \( \mathrm{rank}(F) =  \mathrm{rank}(M) \).
\end{problem}